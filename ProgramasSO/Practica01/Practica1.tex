\documentclass[a4paperx]{article}
\usepackage[utf8]{inputenc}
\usepackage[T1]{fontenc}
\usepackage{float}
\usepackage{fancyhdr}
\pagestyle{fancy}
\fancyhf{}
%\usepackage{pstricks-add}
\pagestyle{empty}

\fancyhead[LE,RO]{\bfseries\thepage}
\fancyhead[LO]{\bfseries\rightmark}
\fancyhead[RE]{\bfseries\leftmark}

\usepackage{amssymb}
\usepackage{amsmath}
\usepackage{amscd}
\usepackage{amsthm}


\renewcommand{\headrulewidth}{0.5pt}
\renewcommand{\footrulewidth}{0pt}
\renewcommand{\thefootnote}{\arabic{footnote}} 
%\renewcommand{\chaptermark}[1]{\markboth{#1}{}}
\renewcommand{\sectionmark}[1]{\markright{\thesection.\ #1}}
%
\fancyhead[LE,RO]{\bfseries\thepage}
\fancyhead[LO]{\bfseries\rightmark}
\fancyhead[RE]{\bfseries\leftmark}


\usepackage{amssymb}
\usepackage{amsmath}
\usepackage{amscd}
\usepackage{amsthm}
%\usepackage[latin1]{inputenc}
%\usepackage[english]{babel}
\usepackage[spanish,mexico]{babel}
\usepackage{enumerate}
\usepackage{pgf,tikz}
\usepackage{makeidx}
\usetikzlibrary{arrows}
\usepackage{graphicx}
\usepackage{float}
\usepackage{pstricks-add}
\usepackage{pgf,tikz}
\usepackage{mathrsfs}
\usetikzlibrary{arrows}
\usepackage[colorlinks=true, linkcolor=blue, urlcolor=red, citecolor=red]{hyperref}
\usepackage{wrapfig}

\usepackage[Glenn]{fncychap}
\ChNameVar{\bfseries\Large}
\ChNumVar{\Huge}
\ChTitleVar{\bfseries\Large}
\ChRuleWidth{0.5 pt}
\ChNameUpperCase
\ChTitleUpperCase
\makeindex


\begin{document}

\title{Pr\'actica 1}
\author{
Aguilar Z\'u\~niga,Gibran 308071087 \and  Alexis Hernández castro 313006636 \and Jesus Martin Ortega Martinez 310183534 \and Daniel Lopez Hernández 309167282 \and Jaime Alberto Martínez López 309256753
}

\maketitle

\begin{abstract}
Introducir al lenguaje C y manejo de apuntadores y memoria.
\end{abstract}

\section{Preguntas}

\textbf{NOTA:}\\

Para ejecutar los programas con Makefile debes pararte en la cartpeta adecuada y poner make para generar el ejecutable, despues ./Programas y se ejecutara el programa correcto. Ejemplo de ejecuci\'on con Makefile.\\


\begin{figure}[H]
\centering
\includegraphics[width=0.65\textwidth]{Buble1.png}
\caption{Ejecuci\'on con MakeFile}
\end{figure}

\begin{figure}[H]
\centering
\includegraphics[width=0.65\textwidth]{Buble2.png}
\caption{Ejecuci\'on con MakeFile}

\begin{figure}[H]
\centering
\includegraphics[width=0.65\textwidth]{Buble3.png}
\caption{Ejecuci\'on con MakeFile}
\end{figure}
\end{figure}

\begin{figure}[H]
\centering
\includegraphics[width=0.65\textwidth]{Buble4.png}
\caption{Ejecuci\'on con MakeFile}
\end{figure}

\begin{enumerate}

\item{Ordenamiento burbuja con apuntadores.}

\begin{enumerate}
\item{Realizar un programa en C, en el cual con cada iteraci\'on del ordenamiento burbuja, se modifiquen las referencias a los datos en memoria, y no los datos contenidos en la misma.}

\begin{figure}[H]
\centering
\includegraphics[width=0.65\textwidth]{BubleSort.png}
\caption{C\'odigo Buble Sort}
\end{figure}

\item{Se deber \'a imprimir en cada iteraci \'on las direcciones de memoria empleando formateadores de cadena especificos para esta acci\'on (\%p)}
\end{enumerate}

\item{Keylogger}

\begin{enumerate}
\item{Empleando la llamada al sistema read en un sistema operativo derivado
de UNIX, interceptar la informaci\'on del dispositivo de entrada $”$teclado$”$
y guardarla en un archivo de texto.}

\item{Las entradas del teclado se se almacenan en estructuras(structs) de
tipo input event, que almacenan entre otros datos el c \'odigo de la
tecla pulsada o liberada, la definici \'on de este tipo de estructuras s
encuentra en sys/types.h y linux/input.h.}

\item{El dispositivo que genera este tipo de estructuras se encuentra de
manera l\'ogica en /dev/input/ en sistemas derivados de UNIX y se
maneja como un evento, el evento cambia entre cada sistema operativo.}
\end{enumerate}


\item{Multiplicaci \'on dinamica de matrices.}\\

\begin{enumerate}

\item{Empleando las funciones malloc(), calloc(), realloc() y free(), crear
arreglos bidimensionales que representen matrices a multiplicar.}\\

\begin{figure}[H]
\centering
\includegraphics[width=0.65\textwidth]{MatricesMalloc.png}
\caption{C\'odigo Matrices Multiplicaci\'on}
\end{figure}

\item{Las dimensiones de las matrices deber\'an ser especificadas por el
usuario, y no se debe emplear mas espacio de almacenamiento que el
requerido por las matrices.}\\

\begin{figure}[H]
\centering
\includegraphics[width=0.65\textwidth]{MatricesDimension.png}
\caption{C\'odigo Matrices Multiplicaci\'on}
\end{figure}

\item{Se tiene la opci\'on de solicitar los elementos de las matrices al usuario o generarlos de manera automatica a trav\'es de funciones de generaci\'on de numeros aleatorios propias de C. En ambos casos de deben imprimir las matrices y el resultado de la multiplicaci \'on.}\\

\begin{figure}[H]
\centering
\includegraphics[width=0.65\textwidth]{EjecucionMatrices.png}
\caption{C\'odigo Matrices Multiplicaci\'on}
\end{figure}

\end{enumerate}

\item{2K38Y}

\begin{enumerate}

\item{Imprimir en pantalla la fecha exacta con segundos para la Ciudad de
M\'exico en la que la forma de medir el tiempo con epoch desbordar\'a
el espacio de almacenamiento valido para un entero de 32 bits.}\\

\begin{figure}[H]
\centering
\includegraphics[width=0.65\textwidth]{2K38YEjecucion.png}
\caption{Ejecuci\'on 2K38Y }
\end{figure}

\item{Restricci\'on. no m\'as de 5 lineas de c\'odigo dentro de la funci\'on main.}\\

\begin{figure}[H]
\centering
\includegraphics[width=0.65\textwidth]{2K38YCodigo.png}
\caption{C\'odigo 2K38Y }
\end{figure}

\end{enumerate}

\end{enumerate}

\end{document}
