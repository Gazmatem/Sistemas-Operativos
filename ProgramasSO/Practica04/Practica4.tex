\documentclass[a4paperx]{article}
\usepackage[utf8]{inputenc}
\usepackage[T1]{fontenc}
\usepackage{float}
\usepackage{fancyhdr}
\pagestyle{fancy}
\fancyhf{}
%\usepackage{pstricks-add}
\pagestyle{empty}

\fancyhead[LE,RO]{\bfseries\thepage}
\fancyhead[LO]{\bfseries\rightmark}
\fancyhead[RE]{\bfseries\leftmark}

\usepackage{amssymb}
\usepackage{amsmath}
\usepackage{amscd}
\usepackage{amsthm}


\renewcommand{\headrulewidth}{0.5pt}
\renewcommand{\footrulewidth}{0pt}
\renewcommand{\thefootnote}{\arabic{footnote}} 
%\renewcommand{\chaptermark}[1]{\markboth{#1}{}}
\renewcommand{\sectionmark}[1]{\markright{\thesection.\ #1}}
%
\fancyhead[LE,RO]{\bfseries\thepage}
\fancyhead[LO]{\bfseries\rightmark}
\fancyhead[RE]{\bfseries\leftmark}


\usepackage{amssymb}
\usepackage{amsmath}
\usepackage{amscd}
\usepackage{amsthm}
%\usepackage[latin1]{inputenc}
%\usepackage[english]{babel}
\usepackage[spanish,mexico]{babel}
\usepackage{enumerate}
\usepackage{pgf,tikz}
\usepackage{makeidx}
\usetikzlibrary{arrows}
\usepackage{graphicx}
\usepackage{float}
\usepackage{pstricks-add}
\usepackage{pgf,tikz}
\usepackage{mathrsfs}
\usetikzlibrary{arrows}
\usepackage[colorlinks=true, linkcolor=blue, urlcolor=red, citecolor=red]{hyperref}
\usepackage{wrapfig}

\usepackage[Glenn]{fncychap}
\ChNameVar{\bfseries\Large}
\ChNumVar{\Huge}
\ChTitleVar{\bfseries\Large}
\ChRuleWidth{0.5 pt}
\ChNameUpperCase
\ChTitleUpperCase
\makeindex


\begin{document}

\title{Pr\'actica 4}
\author{
Aguilar Z\'u\~niga,Gibran 308071087 \and  Alexis Hernández castro 313006636 \and Jesus Martin Ortega Martinez 310183534 \and Daniel Lopez Hernández 309167282 \and Jaime Alberto Martínez López 309256753
}

\maketitle

\begin{abstract}
Algoritmos de Calendarizaci\'on.
\end{abstract}

\section{Preguntas}

\begin{enumerate}

\item{Para esta practica se deben simular los siguientes algoritmos de calendarizaci\'on. FCFS, SJF, y Round Robin(EXTRA).}\\

\item{Cada uno de los programas leer\'a una lista de procesos con datos referentes a los procesos. EL programa simular\'a la ejecucu\'on de los procesos e imprimir\'a el tiempo que le toma a cada proceso completarse (tiempo de ejecuci\'on) y el tiempo de espera, as\'i como calcular el promedio de ejecuci\'on.}\\

\begin{figure}[H]
\centering
\includegraphics[width=0.75\textwidth]{FCFSEjecucion}
\caption{Ejecuci\'on con Makefile de algoritmo FCFSE}
\end{figure}

\item{La entrada puede ser a trav\'es de la linea de comandos o de un archivo
con nombre espec\'ifico en un directorio espec\'ifico.(Los procesos pueden ser
especificados para ser simulados por el programa o generados con datos de
manera aleatoria pero esto resta un punto de la cali\'icaci\'on total de la pr\'actica) Los procesos ser\'an leidos linea a linea, en el orden en el que estos llegar\'ian al sistema operativo. Cada proceso debe tener un nombre, seguido
de el tiempo de llegada del proceso, , seguido del tiempo total de ejecuci\'on, el tiempo transcurrido entre llamadas al sistema(interrupciones),
el tiempo trascurrido en esperas y procesamiento, y finalmente el valor de
prioridad (a menor valor mayor prioridad). }\\

\begin{figure}[H]
\centering
\includegraphics[width=0.75\textwidth]{SFJEjecucion}
\caption{Ejecuci\'on con Makefile de algoritmo SFJE}
\end{figure}

\item{El intervalo de tiempo para los procesos puede ser asignado aleatoria-
mente, pero de ser elegido arbitariamente emplear el valor 3. La salida
debe tener la lista de procesos y el tiempo que tomo a cada uno terminar su ejecuci\'on, adem\'as del tiempo promedio de ejecuci\'on para todos los
procesos en la lista.}\\

\begin{figure}[H]
\centering
\includegraphics[width=0.75\textwidth]{RounRobinEjecucion}
\caption{Ejecuci\'on con Makefile de algoritmo RounRobin}
\end{figure}

\end{enumerate}

\addcontentsline{toc}{chapter}{Bibliografía}
\begin{thebibliography}{99}

\bibitem{Shell} https://labprograms.wordpress.com/2009/07/24/cpu-scheduling-algorithms-fcfs-sjf-and-round-robin/ {\it Fundamentos de sistemas operativos}. 

\bibitem{ShellSort}  https://www.thecrazyprogrammer.com/2014/11/c-cpp-program-for-first-come-first-served-fcfs.html. {\it Fundamentos de sistemas operativos}. 

\end{thebibliography}

\end{document}
