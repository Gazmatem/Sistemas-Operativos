\documentclass[a4paperx]{article}
\usepackage[utf8]{inputenc}
\usepackage[T1]{fontenc}
\usepackage{float}
\usepackage{fancyhdr}
\pagestyle{fancy}
\fancyhf{}
%\usepackage{pstricks-add}
\pagestyle{empty}

\fancyhead[LE,RO]{\bfseries\thepage}
\fancyhead[LO]{\bfseries\rightmark}
\fancyhead[RE]{\bfseries\leftmark}

\usepackage{amssymb}
\usepackage{amsmath}
\usepackage{amscd}
\usepackage{amsthm}


\renewcommand{\headrulewidth}{0.5pt}
\renewcommand{\footrulewidth}{0pt}
\renewcommand{\thefootnote}{\arabic{footnote}} 
%\renewcommand{\chaptermark}[1]{\markboth{#1}{}}
\renewcommand{\sectionmark}[1]{\markright{\thesection.\ #1}}
%
\fancyhead[LE,RO]{\bfseries\thepage}
\fancyhead[LO]{\bfseries\rightmark}
\fancyhead[RE]{\bfseries\leftmark}


\usepackage{amssymb}
\usepackage{amsmath}
\usepackage{amscd}
\usepackage{amsthm}
%\usepackage[latin1]{inputenc}
%\usepackage[english]{babel}
\usepackage[spanish,mexico]{babel}
\usepackage{enumerate}
\usepackage{pgf,tikz}
\usepackage{makeidx}
\usetikzlibrary{arrows}
\usepackage{graphicx}
\usepackage{float}
\usepackage{pstricks-add}
\usepackage{pgf,tikz}
\usepackage{mathrsfs}
\usetikzlibrary{arrows}
\usepackage[colorlinks=true, linkcolor=blue, urlcolor=red, citecolor=red]{hyperref}
\usepackage{wrapfig}

\usepackage[Glenn]{fncychap}
\ChNameVar{\bfseries\Large}
\ChNumVar{\Huge}
\ChTitleVar{\bfseries\Large}
\ChRuleWidth{0.5 pt}
\ChNameUpperCase
\ChTitleUpperCase
\makeindex


\begin{document}

\title{Pr\'actica 6}
\author{
Aguilar Z\'u\~niga,Gibran 308071087 \and  Alexis Hernández castro 313006636 \and Jesus Martin Ortega Martinez 310183534 \and Daniel Lopez Hernández 309167282 \and Jaime Alberto Martínez López 309256753
}

\maketitle

\begin{abstract}
Describir como implementar una unidad de manejo de memoria en un
sistema operativo multiusuario. Simular el algoritmo First Fit, Best Fit de administraci\'on de memoria.
\end{abstract}

\section{Preguntas}

\begin{enumerate}


\item{Describir como implementar una unidad de manejo de memoria en un
sistema operativo multiusuario.}\\

\item{Simular el algoritmo First Fit, Best Fit de administraci\'on de memoria.}\\


\item{Seguir los siguientes pasos para la implementaci\'on, teniendo en cuenta
que es una simulaci\'on, por lo que todo deber\'ia ser provisto a trav\'es de un
archivo de configuraci\'on o preguntando al usuario.}\\

\begin{figure}[H]
\centering
\includegraphics[width=0.75\textwidth]{FirstFitEjecucion}
\caption{Ejecucion del algoritmo FirstFit}
\end{figure}

\begin{enumerate}

\item{Obtener el tamano del segmento, el numero de archivos a ser alojados y su tamano correspondiente. (N\'umero de bloques, n\'umero de
archivos)}\\

\item{Comprobar si es posible emplear ambos algoritmos con los datos
proporcionados. Dando como salida un resultado exitoso, o si los
recursos son insuficientes, manteniendo una salida lo mas descriptiva
posible sobre la ejecuci\'on.}\\

\item{Para el algoritmo first fit, alojar el proceso al primer segmento libre,
y marcarlo como ocupado para que no pueda ser ocupado.}\\

\item{Para el algoritmos best fit, se deben ordenar los segmentos por tama\~no, y se debe alojar el proceso en el segmento que sea igual o de mayor
tama\~no que el proceso y marcarlo para que no pueda ser usado.}\\

\item{Para este programa es posible utilizar arreglos o bien, memoria din\'amica.}\\

\end{enumerate}

\item{Peluqueria}\\

\begin{enumerate}

\item{Asuma que una peluqueria que usted visita tiene tres sillas, y el
area de espera tiene 4 revistas y una cantidad ilimitada de libros de
sistemas operativos para otros clientes. Debido a los problemas de
espacio el numero total de clientes en la peluqueria esta restringido
a 10.}\\

\begin{figure}[H]
\centering
\includegraphics[width=0.75\textwidth]{BestFitEjecucion}
\caption{Ejecucion del algoritmo FirstFit}
\end{figure}

\item{Un cliente no debe entrar a la peluqueria si esta al limite de su
capacidad. Una vez dentro, si no hay un peluquero libre, el cliente
toma una revista, y si no estan disponibles, toma un libro.}\\

\item{Una vez que un peluquero este libre, uno de los clientes que haya
leido una de las revistas por mas tiempo, es quien puede pasar a la
silla, y si hay clientes leyendo un libro de sistemas operativos, aquel
que haya leido la mayor cantidad de tiempo, puede tomar la revista.}\\

\item{Cuando un corte de cabello se nalice, cualquier peluquero puede
aceptar el pago, pero ya que solo se tiene una caja, solo se puede
aceptar un pago a la vez.}\\

\item{Los peluqueros dividen su tiempo entre cortar el cabello, aceptar
pagos, y durmiendo en su silla esperando por clientes(Ellos no leen
revistas ni libros de sistemas operativos).}\\

\item{La llegada de un cliente puede ser simulada a trav\'es del teclado(por
ejemplo, al presionar l), el hilo principal puede iniciar el hilo de la
llegada de un nuevo cliente.}\\

\item{Se puede asumir que un maximo de 50 clientes va a llegar.}\\

\item{Utilizar tiempos aleatorios en el termino del corte.}\\

\item{Salida esperada: Se deben imprimir las acciones de la siguiente manera:
Peluquero 1: descansando\\
Lee revista: Cliente n\\
Con peluquero1: cliente n\\
Paga: clienten\\
cobra peluquero2}\\

\item{El formato es decisi\'on del programador.}\\

\begin{figure}[H]
\centering
\includegraphics[width=0.75\textwidth]{PeluqueriaEjecucion}
\caption{Ejecucion del algoritmo FirstFit}
\end{figure}

\end{enumerate}

\end{enumerate}

\addcontentsline{toc}{chapter}{Bibliografía}
\begin{thebibliography}{99}

\bibitem{Shell} https://www.ibm.com/support/knowledgecenter/en/SSLTBW\\
\_2.3.0/com.ibm.zos.v2r3.bpxbd00/rtrea.htm. {\it Fundamentos de sistemas operativos}. 

\bibitem{ShellSort}  https://www.geeksforgeeks.org/signals-c-set-2/. {\it Fundamentos de sistemas operativos}. 

\end{thebibliography}

\end{document}
